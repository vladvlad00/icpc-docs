Let $N(x)$ and $D(x)$ be polynomial functions of $x$.
We can break down $N(x)/D(x)$ using partial fraction expansion.
First,
if the degree of $N$ is greater than or equal to the degree of $D$,
divide $N$ by $D$,
obtaining
$$
{N(x) \over D(x)} = Q(x) + {N'(x) \over D(x)},
$$
where the degree of $N'$ is less than that of $D$.
Second,
factor $D(x)$.
Use the following rules:
For a non-repeated factor:
$$
{N(x) \over (x-a) D(x)} = {A \over x-a} + {N'(x) \over D(x)},
$$
where
$$
A = \left[{N(x) \over D(x)}\right]_{x=a}.
$$
For a repeated factor:
$$
{N(x) \over (x-a)^m D(x)} = \sum_{k=0}^{m-1}{A_k \over (x-a)^{m-k}} + {N'(x) \over D(x)},
$$
where
$$
A_k = {1 \over k!}\left[{d^k \over dx^k} \left({N(x) \over D(x)}\right)\right]_{x=a}.
$$
